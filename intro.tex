
\chapter{Pengenalan}
\section{Pengenalan}
\label{pengenalan}

\LaTeX{} merupakan satu perisian ``typesetting" yang dicipta oleh Leslie Lamport. \LaTeX{} berasal dari perisian \TeX{} yang ditulis oleh Donald Knuth, di mana Knuth tidak berpuas hati
dengan mutu font perisian pemprosesan perkataan sewaktu itu.

\subsection{Platform}
\LaTeX{} boleh digunakan, di antaranya di dalam sistem operasi berikut:

\begin{itemize}
\item Microsoft Windows (menggunakan WinEdt, LEd dan Lyx)
\item GNU/Linux (menggunakan \index{penyunting!Kile}Kile, \index{penyunting!Vim}VIM, \index{penyunting!Emacs}Emacs dan lain-lain penyunting)
\item Mac OS
\end{itemize}

%\paragraph
Pemilihan \index{penyunting|textbf} yang digunakan bergantung kepada citarasa pengguna, dan ia adalah sangat subjektif. Seperti saya sendiri, kadang-kadang saya menggunakan Kile dan kadang-kadang hanya 
menggunakan perisian ringan VIM. 

\subsection{Sokongan}
\LaTeX{} mempunyai peminat dan penyokongnya yang tersendiri, terdiri daripada khalayak yang menggunakannya secara intensif. Kebiasaannya, soalan teknikal berkaitan \LaTeX{} dibincangkan di dalam
mailing list ataupun forum-forum di Internet. 

\section{Sampel yang dihasilkan menggunakan \LaTeX}
\LaTeX{} banyak digunakan samada oleh pelajar-pelajar universiti yang menyiapkan laporan projek, tesis ataupun artikel ataupun mereka yang berkecimpung di dalam bidang penulisan.


\section{Font yang disokong oleh \LaTeX}
\label{sokongan}
\latex{} menyokong penggunaan font \index{Arab} Arab dan \index{Jawi} Jawi, selain daripada huruf Roman\footnote{memandangkan buku ini ditulis untuk pembaca berbahasa Melayu}.
Sebagai contoh untuk font Arab;\\\\


\vocalize
\arabtrue
\begin{RLtext}
\large{
al-salAm `alaykum
\setnash{al-salAm `alaykum,}
\setnashbf{al-salAm `alaykum,}
\setnastaliq{al-salAm `alaykum}
}
\end{RLtext}

\hspace{1cm}

Dan font Jawi;\\

\novocalize
\setmalay
<cUbA lIht tUlIsn inI,>
{\color{blue}\setnashbf{<cUbA lIht tUlIsn inI,>}}
{\color{red}\setnastaliq{<cUbA lIht tUlIsn inI,>}}\\\\

Selain itu, \latex{} juga boleh menggunakan pakej yang ditetapkan sendiri oleh pengguna (user-customized). 



\section{Kenapa guna \LaTeX?}
Ada beberapa sebab kenapa anda perlu mempertimbangkan untuk menggunakan \LaTeX{}, di antaranya ialah:

\begin{itemize}
\item sokongan perisian percuma, atau sekiranya anda mampu anda boleh membeli perisian komersial untuk membantu penulisan anda
\item sokongan Bib\TeX{}, satu perisian yang membantu anda untuk mengatur letak \mbox{``}citation\mbox{''} pada penulisan anda
\item susun atur nombor secara automatik, di mana anda tidak perlu risau tentang atur letak kepala dokumen anda (header)
\item diterima sebagai satu piawaian (standard) sekiranya anda ingin menghantar artikel ataupun jurnal ke mana-mana seminar antarabangsa (sekiranya dinyatakan)
\end{itemize}

\section{Mendapatkan pakej \latex}

\subsubsection{Windows}

Sekiranya anda menggunakan Microsoft Windows, anda boleh dapatkan pakej \latex{} yang terkandung secara pukal di dalam penginstal (berekstensi .exe).\\

\begin{itemize}

\item WinEdt\cite{winedt} 
\item Lyx\cite{lyx} 
\item LEd (Latex Editor) \cite{led}

\end{itemize}


\subsubsection{Linux}

Sekiranya anda menggunakan Ubuntu Linux dan menggunakan Kile\cite{kile}; \\

\begin{Verbatim}[frame=single]
apt-get install kile
\end{Verbatim}

dan semua kebergantungan(dependencies) akan diuruskan oleh pengurus pakej apt-get tersebut.
Sesetengah masalah pakej contohnya arabtex dan alqalam boleh dicari sekiranya anda menggunakan ``apt-cache search"

Contohnya, pakej arabtex terkandung di dalam texlive-lang-arab.


