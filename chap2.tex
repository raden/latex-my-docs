\chapter{Format Tulisan}

\section{Pelbagai jenis format tulisan}

Default:
\begin{verse}
Pulau Pandan Jauh ke Tengah \\
Gunung Daik Bercabang Tiga\\
Hancur Badan Dikandung Tanah\\
Budi yang Baik Dikenang Juga\\
\end{verse}
\bigskip
Bold:
\textbf{\begin{quote}
Pulau Pandan Jauh ke Tengah \\
Gunung Daik Bercabang Tiga\\
Hancur Badan Dikandung Tanah\\
Budi yang Baik Dikenang Juga\\
\end{quote}}
\bigskip
Italics:
\textit
{\begin{quote}
Pulau Pandan Jauh ke Tengah \\
Gunung Daik Bercabang Tiga\\
Hancur Badan Dikandung Tanah\\
Budi yang Baik Dikenang Juga\\
\end{quote}
}
\bigskip
Untuk gaya \index{font}font selain di atas, secara asasnya kita gunakan:\\

\begin{Verbatim}[frame=single]
\textbf{teks anda di sini} % ini untuk bold
\textit{teks anda di sini} % ini untuk italics
\texttt{teks anda di sini} % ini untuk teletype 
\textrm{teks anda di sini} % ini untuk Roman
\textsf{teks anda di sini} % ini untuk serif
\textup{teks anda di sini} % ini untuk TitleCase
\textsl{teks anda di sini} % ini untuk font senget ke kanan
\textsc{teks anda di sini} % ini untuk font CAPS kecil
\textmd{teks anda di sini} % ini untuk font pertengahan, antara normal dan bold
\end{Verbatim}

Kita lihat contoh berikut untuk campuran font ini:\\

{\begin{quote}
\textsf{Pulau Pandan} \textrm{Jauh ke Tengah} \\
\textup{Gunung Daik} \textsl{Bercabang Tiga}\\
\textsc{Hancur Badan} \textmd{Dikandung Tanah}\\
\textbf{Budi yang Baik} \textit{Dikenang Juga}\\
\end{quote}
}

di mana ia sebenarnya adalah :\\

\begin{Verbatim}[frame=single]

{\begin{quote}
\textsf{Pulau Pandan} \textrm{Jauh ke Tengah} \\
\textup{Gunung Daik} \textsl{Bercabang Tiga}\\
\textsc{Hancur Badan} \textmd{Dikandung Tanah}\\
\textbf{Budi yang Baik} \textit{Dikenang Juga}\\
\end{quote}
}
\end{Verbatim}