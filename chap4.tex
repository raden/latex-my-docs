\chapter{Penulisan artikel}

\section{Artikel komputer}
Sekiranya anda terlibat di dalam penulisan artikel saintifik, ada kemungkinan di mana anda perlu memasukan kod sumber anda di dalam artikel anda. 
Kita lihat contoh kod yang ditulis menggunakan bahasa Python di bawah:
\bigskip

\begin{minipage}{\textwidth}
\begin{lstlisting}[frame=shadowbox,language=python]
print "Hello World!"
\end{lstlisting}
\captionof{figure}{Hello World ringkas tanpa warna}
\label{hello-simple}
\end{minipage} \\

Di samping itu, anda juga boleh menggunakan fungsi penyerlahan sintaks \textit{(syntax highlighting)} seperti tertera berikut:\\

\begin{minipage}{\textwidth}
\begin{Verbatim}[frame=single,commandchars=@\[\]]
@PYay[print] @PYaB["]@PYaB[Hello, World!]@PYaB["] 
\end{Verbatim}
\captionof{figure}{Hello World ringkas dengan warna}
\label{hello-colour}
\end{minipage} \\

Kita tengok contoh yang lain yang lebih panjang kodnya.

\begin{minipage}{\textwidth}
\begin{Verbatim}[frame=single,commandchars=@\[\]]
@PYbc[import] @PYaV[gettext]
gettext@PYbe[.]bindtextdomain (@PYaB[']@PYaB[piton]@PYaB['],@PYaB[']@PYaB[/usr/share/locale]@PYaB['])
gettext@PYbe[.]textdomain(@PYaB[']@PYaB[piton]@PYaB['])
_@PYbe[=] gettext@PYbe[.]gettext
@PYay[print] _(@PYaB[']@PYaB[python adalah mudah]@PYaB['])
@PYay[print] _(@PYaB[']@PYaB[semudah ini]@PYaB['])
\end{Verbatim}
\captionof{figure}{Kod lebih panjang dengan warna}
\label{piton}
\end{minipage} \\

dalam contoh \ref{hello-simple}, \ref{hello-colour} dan \ref{piton} di atas, arahan \emph{verbatim} digunakan supaya set arahan itu tidak dilaksanakan oleh sistem, 
sebaliknya dipaparkan ke dalam skrin.

\section{Artikel dengan formula matematik}
\latex{} boleh membantu anda menyiapkan kerja-kerja yang memerlukan anda memasukkan formula \index{matematik}matematik, termasuklah penulisan saintifik, kertas soalan, mahupun peberbitan lain. 



\subsection{Contoh formula matematik}

\subsubsection{\index{Punca kuasa}Punca kuasa}
Katakan anda ingin memasukkan punca kuasa dua di dalam dokumen anda.

$\sqrt{4} = 2$

dalam perkara ini, kebiasaannya sintaks yang ingin ditukar kepada formula matematik diapit dengan tanda \mbox{\$}.

Contohnya untuk menghasilkan formula tadi, kita tulis seperti berikut:\\

\begin{Verbatim}[frame=single]
 $\sqrt{4} = 2$
\end{Verbatim}

Di mana, dalam contoh di atas, oleh kerana kita ingin membuat punca kuasa dua, iaitu $\sqrt{4}$, maka kita letakkan nombor 4 di dalam kurungan tersebut \mbox{\{4\}}.
Bagaimana sekiranya punca kuasa tiga, empat, dan seterusnya?\\

$\sqrt[3]{8} = 2$\\

dihasilkan dengan;\\

\begin{Verbatim}[frame=single]
 $\sqrt[3]{8} = 2$
\end{Verbatim}

dan,\\

$\sqrt[4]{16} = 2$\\

dengan\\

\begin{Verbatim}[frame=single]
 $\sqrt[4]{16} = 2$
\end{Verbatim}


\subsubsection{Kuasa}
Untuk kuasa, maka seperti contohnya $x^y$, maka ia adalah seperti berikut:\\

\begin{Verbatim}[frame=single]
 $x^y$
\end{Verbatim}


%Pakej matematik di dalam \latex{} ialah menggunakan pakej \nomenclature{AMS}{American Mathematical Society} AMS.
%Untuk itu, apa yang perlu anda lakukan ialah meletakkan \\

%\begin{Verbatim}[frame=single]
% \include{ams}
%\end{Verbatim}

%pada kepala dokumen anda.